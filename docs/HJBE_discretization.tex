\documentclass[11pt]{etk-article}
\usepackage{pstool} 
\usepackage{etk-bib}
\pdfmetadata{}{}{}{}
\externaldocument[FD:]{operator_discretization_finite_differences}

\begin{document}
\title{Solving HJBE with Finite Differences}
\author{Jesse Perla\\UBC}
\date{\today}
\maketitle
 Here, we expand on details for how to discretize HJBE with a control of the drift.\footnote{See \url{operator_discretization_finite_differences.pdf} for more details on the discretization of a linear diffusion operator with finite-differences, and general notation (with equation numbers in that document prefaced by \textit{FD}).}  In particular, this will start by solving the neoclassical growth model with a deterministic and then stochastic TFP.
 
 \section{Neoclassical Growth}
 This solves the simple deterministic neoclassical growth model.
 \subsection{Value Function}
 
 Take a standard neoclassical growth model with capital $k$, consumption $c$, production $f(k)$, utility $u(c)$, depreciation rate $\delta$, and discount rate $\rho$.\footnote{
 This builds on \url{http://www.princeton.edu/~moll/HACTproject/HACT_Additional_Codes.pdf} Section 2.2, with code in \url{http://www.princeton.edu/~moll/HACTproject/HJB_NGM_implicit.m}}
The law of motion for capital in this setup is,
\begin{align}
	\D[t] k(t) &= f(k) - \delta k - c\label{eq:lom-capital}\\
	\intertext{With this, the standard HJBE for the value of capital $k$ is,}
	\rho v(k) &= \max_{c}\set{u(c) + \left(f(k) - \delta k - c\right)v'(k)}\label{eq:HJBE-neoclassical-growth}
	\intertext{Assume an interior $c$ and envelope conditions, then taking the FOC of then the FOC of the choice is,}
	u'(c) &= v'(k)
	\intertext{If we assume a functional form for the utility, such as $u(c) = \frac{c^{1-\gamma}}{1-\gamma}$ then this can be inverted such that}
	c &= \left(v'(k) \right)^{-\frac{1}{\gamma}}\label{eq:c-neoclassical-growth}
	\intertext{And,}
	u(c) &= \frac{\left(v'(k)\right)^{-\frac{1-\gamma}{\gamma}}}{1-\gamma}\label{eq:u-c-neoclassical-growth}
	\intertext{If \cref{eq:c-neoclassical-growth,eq:u-c-neoclassical-growth} was substituted back into \cref{eq:HJBE-neoclassical-growth}, we would have a nonlinear ODE in just $k$}
\end{align}
 

\bibliography{etk-references}

\end{document}