\documentclass[11pt]{etk-article}
\usepackage{pstool} 
\usepackage{etk-bib}
\pdfmetadata{}{}{}{}
\externaldocument[FD:]{operator_discretization_finite_differences}

\begin{document}
\title{Optimal Stopping Problems}
\date{\today}
\maketitle
These notes expand on Ben Moll's superb notes in \url{http://www.princeton.edu/~moll/HACTproject/} on solving option value problems as HJB Variational Inequalities, as discussed \cite{HuangPang1998}.

\section{Optimal Stopping of a Univariate, Time-Homogenous Process}
This section outlines the general approach to the problems.\footnote{See \url{http://www.princeton.edu/~moll/HACTproject/option_simple.pdf} for the simplest case.  The algebra expands on \url{http://www.princeton.edu/~moll/HACTproject/HACT_Numerical_Appendix.pdf} and \url{http://www.princeton.edu/~moll/HACTproject/HACT_Additional_Codes.pdf}.}
\subsection{Variational Inequality Formulation}
To set notation:
\begin{itemize}
	\item $x$ is a stochastic process, with infinitesimal generator $\mathcal{A}$.\footnote{For example, the infinitesimal generator of the SDE $\diff x_t = \mu(x_t)\diff t + \sigma(x_t)\diff\mathbb{W}_t$ is  $\mathcal{A} \equiv \mu(x)\D[x] + \frac{\sigma(x)^2}{2}\D[xx]$}
	\item An agent with state $x$ can optimally stop with value $S(x)$ at any time.
	\item The agent gains utility flow $u(x)$ and discounts the future at rate $\rho > 0$.
\end{itemize}


\paragraph{Classical Formulation}
The typical formulation of this is as a free-boundary value problem.  Assume that the agent chooses an optimal $\hat{x}$ stopping rule, then it will solve the ODE in the continuation region along with value matching and optimal stopping.  That is, find a $v(x)$ continuation value and stopping point $\hat{x}$ that fulfills,\footnote{Note that if anything was a function of time, this would no longer be time-invariant.  For example, if $S(t,x)$, then $\mathcal{A}$ would have a $\D[t]$ term as well, and $\hat{x}(t)$ is the stopping rule at any time}
\begin{align}
	\rho v(x) &= u(x) + \mathcal{A} v(x)\\
	v(\hat{x}) &= S(\hat{x})\\
	v'(\hat{x}) &= 0
\end{align}	
Additionally another boundary value would be required for either a large $\bar{x}$ or a transversality condition.  This formulation requires that there is only a single stopping point.

\paragraph{HJB Variational Inequality Formulation}
Another formulation is to find a $v(x)$ such that the following holds,
\begin{align}
	0 &= \min\set{\rho v(x) - u(x) - \mathcal{A} v(x),\, v(x) - S(x) }\label{eq:HJB-variational-simple-operator}
\end{align}	
for all $x$ (adding an appropriate boundary conditions $x$).  This is a variational inequality since $\mathcal{A}$ is a differential operator.  While the value matching condition is clear here, it can be shown that this implies the smooth pasting condition.


\subsection{Discretized Problem}\label{sec:discretized-stationary}
For an arbitrary operator $\mathcal{A}$ with associated boundary values, we will find a solution to \cref{eq:HJB-variational-simple-operator} by discretizing the $\mathcal{A}$ operator on a grid $\set{x_i}_{i=1}^I$ with $x_1 = \underline{x}$ and $x_I = \bar{x}$ subject to appropriate boundary values on $\underline{x}$ and $\bar{x}$.  See \url{operator_discretization_finite_differences.pdf} notes for details on the discretization of operators.
\paragraph{Linear Operator}
In the case of a linear operator $\mathcal{A}$, the resulting discretized finite-difference operator is a matrix $A$.  Denote $v \equiv \set{v(x_i)}_{i=1}^I, u \equiv \set{u(x_i)}_{i=1}^I$, and $S \equiv \set{s(x_i)}_{i=1}^I$, then \cref{eq:HJB-variational-simple-operator} becomes
\begin{align}
	0 &= \min\set{\rho v - u - A v,\, v - S }\label{eq:HJB-variational-simple-discrete}
\end{align}
where $A\in\R^{I\times I}$.  A solution to this problem is a $v$ fulfilling \cref{eq:HJB-variational-simple-discrete}.  Note that the boundary values are already in the $A$ operator and do not need to be discussed separately.

\subsection{LCP Formulation}\label{sec:LCP-formulation}
Following \url{http://www.princeton.edu/~moll/HACTproject/option_simple.pdf} with only minor variations, we note that \cref{eq:HJB-variational-simple-operator} is a linear-complementarity problem. 
Define the following (given the identity matrix $\mathbf{I}\in\R^{I\times I}$),
\begin{align}
	B &\equiv \rho \mathbf{I} - A\label{eq:B}\\
	z &\equiv v - S\label{eq:z}\\
	q &\equiv - u + B S\label{eq:q}\\
		w &\equiv B z + q\label{eq:w}\\
	\intertext{Substitute \cref{eq:B,eq:z,eq:q} into \cref{eq:HJB-variational-simple-discrete}}
	0 &= \min\set{B z + q,\, z }
	\intertext{Which could be written as complementarity slackness conditions,}
	z^T(B z + q) &= 0 \\
	z &\geq 0\\
	B z + q &\geq 0	\\
	\intertext{Which can be written succinctly with complementarity constraints as,}
	0 &\leq (B z + q) \perp z \geq 0
\end{align}
Alternatively, some LCP and MCP solvers prefer to have a slack variable introduced.  Use \cref{eq:w} to find the LCP problem as finding $w,z\in\R^I$ such that,
\begin{align}
		w &\equiv B z + q\label{eq:w-equality-constraint}\\
		0 \leq w &\perp z \geq 0
\end{align}
where \cref{eq:w-equality-constraint} is added to the solver with a linear equality constraint.

In either case, to unpack the results, drop any slack variables to get the $z$ vector, and undo the transformation in \cref{eq:z}.

\section{Optimal Stopping of a Univariate, Time-Varying Process}



\subsection{Discretizing the Stochastic Process}
Assume $t \in [0, T]$, where $T$ may be a very large number if the system converges. 
\paragraph{Stopping}
The value of stopping at any time $t$ with state $x$ is now $S(t,x)$.  Assume that $S(t,x)$ converges so that $S(t,x) = S(T,x)$ for all $t \geq T$.

\paragraph{Discretizing the Operator}
Here, we follow \cref{FD:sec:time-dependent-univariate-diffusion}.  To summarize, 
 The SDE for the state is now \begin{align}
\diff x_t = \mu(t, x_t)\diff t + \sigma(t, x_t)\diff\mathbb{W}_t\label{eq:x-t-SDE}
\end{align}
and the infinitesimal generator is 
\begin{align}
\mathcal{A} &\equiv \mu(t,x)\D[x] + \frac{\sigma(t,x)^2}{2}\D[xx] + \D[t]\label{eq:A-generator-univariate-diffusion-t}
\end{align}
Even if the $\mu(\cdot)$ and $\sigma(\cdot)$ are independent of time, the discretized generator used for solving the problem may be dependent on time due to boundary values, etc.

\paragraph{Boundary Values}
As before, we will start by assuming reflecting boundaries at some $\underline{x}$ and $\bar{x}$, so that the boundary values are,
\begin{align}
\D[x]v(t,\underline{x}) &= 0\\
\D[x]v(t,\bar{x}) &= 0\\
\intertext{Furthermore, at time $T$ the system is in a steady state, and we can use}
\D[t]v(t,x) &= 0,\, \text{ for all } t \geq T\label{eq:D-steady-state-T}
\end{align} 

\subsection{Variational Formulation}
The optimal stopping problem can then be cast as finding a $v(t,x)$ such that for all $t\in[0,T]$ and $z\in[\underline{x},\bar{x}]$,
\begin{align}
0 &= \min\set{\rho v(t,x) - u(t,x) - \mathcal{A} v(t,x),\, v(t,x) - S(t,x) }\label{eq:HJB-variational-simple-operator-time-varying}
\end{align}	
Note that this is in nearly the same form as \cref{eq:HJB-variational-simple-operator}.
\subsection{Discretized Problem}
Following \cref{sec:discretized-stationary}, we will discretize the $\mathcal{A}$ operator, $v$, etc. to convert \cref{eq:HJB-variational-simple-operator-time-varying} into a linear complementarity problem.

First, discretize $u\in\R^{N I}$ and $S\in\R^{N I}$ on the grid of $t_n$ and $x_i$,
\begin{align}
u^n_i &\equiv u(t_n, x_i)\\
S^n_i &\equiv S(t_n, x_i)\\
\intertext{Stack these up to form a vector for a given time period.  For example,}
S^n &\equiv \begin{bmatrix} S^n_1 &  S^n_2 & \ldots & S^n_I\end{bmatrix}^{\intercal}\in \R^I
\intertext{Then stack these up over time,}
S &\equiv \begin{bmatrix} S^{1\intercal} &  S^{2\intercal} & \ldots&  S^{N\intercal}\end{bmatrix}^{\intercal}\in \R^{N I}
\end{align}
and equivalently stacked up for $u$.  Then, use the same discretization of $\mathcal{A}$ as given in from \cref{FD:sec:time-dependent-univariate-diffusion} to $A$, 
With this, the discretized version has the identical form to that of \cref{eq:HJB-variational-simple-discrete}
\begin{align}
0 &= \min\set{\rho v - u - A v,\, v - S }
\end{align}
Which can be solved via the same mapping to an LCP problem described in \cref{sec:LCP-formulation}


\bibliography{etk-references}

\end{document}